\chapter*{What Is Fedora About}
\section*{Who Is It For?}

Fedora is available in several different editions, each targeted at different kinds of users. The \emph{Fedora~Workstation} edition this handbook is about is designed for users who use computers primarily to create, from developers to graphic designers, musicians, and writers.

\emph{Fedora~Workstation} features the~\emph{GNOME~3} environment and a variety of tools for developers (such as \emph{Builder}), applications for virtualization (\emph{Boxes}), container management tools (\emph{Docker}), and many more. But it is definitely not just for developers and engineers! It~also comes with several tools for video editing (\emph{PiTiVi}), audio editing (\emph{Audacity}), as well as for editing bitmap (\emph{GIMP}), vector (\emph{Inkscape}), and 3D (\emph{Blender}) graphics.

In a sense, \emph{Fedora~Workstation} is a better operating system for an average user than it ever was.

\section*{Open Source and Freedom}

Open source software is software that has made its source code available for distribution, modification, and use for any purpose. Fedora has been created 15 years ago and is maintained by community of~professional developers and volunteers who are passionate about open source software, and because of this, it has always included and will continue to include only open source software. It doesn’t include any software that can’t be reviewed at the source code level.

At the same time, Fedora doesn't prevent you from installing any software you want. Do you want a non-open source application, such as \emph{Google Chrome}? You can easily install it. But, even without 3rd-party and non-open source software, you'll find more then 20,000 software packages available in Fedora, representing various applications, extensions, and libraries. Many of these open source programs are not just alternatives, but are often better or more powerful than their closed source counterparts. For more information, see the \emph{Installing New Software} chapter. %TODO: dopis cislo stranky

Fedora is, however, not just about open source, but is also passionate about freedom in the form of software licenses and patents. Fedora respects software patent law (even though it may not be relevant in~some parts of the world). You can be sure that you are not breaking laws by using Fedora.

\section*{Security and Privacy}

Fedora development prioritizes security and includes clear rules to ensure that it remains its primary focus. Like any other large-scale software project, each release is carefully tested. Moreover, Fedora is the base for Red~Hat~Enterprise~Linux, a commercially supported operating system offered by Red~Hat. Red~Hat employs a lot of developers, many of whom are involved in the \emph{Fedora Project}, and owns Fedora trademarks.

The Fedora community support lasts for two release cycles plus one month. This means that with a new release every 6 months, each version of Fedora is supported for 13~months. The Fedora community support includes security fixes and kernel updates, which is significant because new kernels mean improved and new hardware support. This is a big advantage of using Fedora.

The Fedora community carefully selects the software to include in~the operating system and is concerned about privacy. Therefore, Fedora doesn't include any programs that send any sensitive data without your permission. Fedora doesn't spy on you, it respects your privacy.

\section*{Leading the Way}

Fedora integrates a lot of new software and is often the first (or one of~the first) operating systems to adopt new technology. Fedora is very often where new technologies are being developed and tested.

Open source software is about collaboration and people who are involved in the \emph{Fedora Project} are very often active in other projects, too. They don't wait until someone else develops software so they can include it in Fedora---instead, they actively participate in the project's development and collaborate with participants from other Linux operating systems or independent developers. It is common that such software goes on to become the de facto standard of the Linux world.

Fedora users keep fingers on the pulse of innovation and change and are ahead of others. This is why it is easy to say that Fedora is progressive, innovative, and leading the way.
\endinput
