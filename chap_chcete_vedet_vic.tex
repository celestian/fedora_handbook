\chapter*{Chcete vědět víc?}
\section*{Kam se podívat dál?}
Naše příručka má skromné ambice, co když ale narazíte na problém, který nejste schopni rychle vyřešit? Základní pravidlo zní: problém, který řešíte, už takřka jistě řešil někdo před vámi. Kam se tedy podívat?
\begin{enumerate}
\item \url{fedora.cz} -- z~českých webů ve vztahu k~Fedoře zásadní stránky (a univerální rozcestník). Obsahuje množství článků, tipů a dalších informací.
\item \url{wiki.fedora.cz} -- přírzučka o~Fedoře, na které mnoho lidí na ní odvedlo obrovské množství práce. Mnoho návodů, které hledáte, budou právě zde.
\item \url{forum.fedora.cz} -- fórum v~českém jazyce hodící se vždy, když nejste schopni problém vyřešit sami, nebo když sami chcete nabídnout svou pomoc.
\item \url{fedoraforum.org} -- fórum v~anglickém jazyce s~rozsáhlou uživatelskou základou a dlouhou historií řešených problémů. Nyní už byste se mohli být v~cíli.
\item \url{fedoramagazine.org} -- články o~dění kolem Fedory. Nové aplikace a oznámení.
\end{enumerate}
\section*{Co když narazím na chybu?}
Je docela možné, že se dostanete do situace, kdy narazíte chybu zcela novou. Co pak? Fedora používá bugzillu společnosti Red Hat, která je dostupná z~adresy bugzilla.redhat.com. (Nutno podotknout, že nezbytností při hlášení chyb je angličtina, alespoň na základní úrovni.) Není to ale jediná možnost, jak chybu ohlásit. Ve Fedoře je dostupný nástroj ABRT. Budeme-li citovat klasický snímek, pak daný nástroj vystihuje spojení \uv{každý přispívá pod svých možností} a ABRT koná přesně v~tom duchu.

\section*{Další edice Fedory}
Zde se dostáváme k~věcem, ve kterých se Fedora odlišuje od obecného spektra linuxových distribucí. Fedora doznala své podoby ve třech hlavních edicích. Už dříve popsaná je edice Workstation, dále pak tu máme edice Server a Cloud. Pojďme na ně. Jsou to možná pojmy, které by si člověk bez dalšího (minimálně ještě v~nedávné době) s~Fedorou nezbytně nespojoval. Nicméně nyní to je jedna z~priorit a odpovídají tomu i do distribuce zahrnuté nástroje. U~edice Server lze vyzdvihnout aplikace jako Cockpit sloužící pro vzdálenou správu běžících serverů prostřednictvím webového prohlížeče, nebo správu serverových rolí přes Rolekit. U~edice Cloud pak máme k~dispozici minimalistickou verzi Fedory, která právě tak akorát umožňuje nasazení kontejnerů. A~že existuje hned několik obrazů Fedory Cloud připravených pro nasazení v~prostředí OpenStack, VirtualBox a dalších.

\section*{Fedora a spiny}
Vše, co dosud padlo ve vztahu k~Fedora Workstation, se týkalo výchozího cílení distribuce, jejímž desktopovým prostředím je GNOME~3 a jeho GNOME Shell. Co když se ale chceme rozhlédnout dále? Ve Fedoře existují tzv. spiny, čili připravené instalační obrazy se specifickým cílením. Jste zvyklí na KDE Plasma? Je tu KDE Plasma Desktop. Xfce? LXDE? Pro vše tu jsou připravené obrazy. Ale ani u~Fedora spinů se nemusíme zastavit. V~dnešní době je velmi aktuální procesorová architektura ARM (nejčastější mobilní platforma, platforma pro různé vývojové desky à la Banana Pi, BeagleBone, nebo v~případě Fedora Remixu i Raspberry Pi). Stejně jako některé další linuxové distribuce, i Fedora má pro tuto architekturu připravené řešení ve formě jak serverové architektury, tak minimálního instalačního obrazu pro obecné použití.
