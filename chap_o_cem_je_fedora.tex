\chapter*{O~čem je Fedora}
\section*{Pro koho je?}
Fedora v~edici \emph{Workstation} cílí na uživatele, jejichž primárním zájmem je tvůrčí práce, ať už se jedná o~vývoj, nebo jiné činnosti. Nabízí odladěné prostředí \emph{GNOME~3} a spektrum aplikací, jako jsou pro vývojáře nástroje \emph{DevAssistant} nebo nově \emph{Builder}, různých aplikací pro virtualizaci (\emph{Boxes)}, správů kontejnerů (\emph{Docker}) a dalších. Znamená to, že Fedora rezignuje na běžného uživatele a opouští ho? Ne. Fedoru lze stejně tak dobře použít i pro multimediální účely, mezi které můžeme počítat i střih videa (\emph{PiTiVi}), práci s~bitmapovou (\emph{GIMP}) a vektorovou (\emph{Inkscape}) grafikou. V~určitém smyslu je tak Fedora lepší distribucí i pro běžného uživatele než byla kdy dříve. 

\section*{Otevřenost}
Fedora je komunitní linuxová distribuce s~více než desetiletou tradicí. Vždy obsahovala a bude obsahovat jen svobodný, čili takový software, který je open source a může být volně šířen, upravován a používán za libovolným účelem. Fedora neobsahuje software, na který byste si nemohli \uv{sáhnout} na úrovni zdrojového kódu. Zároveň ale nijak neomezuje software, který si do systému nainstalujete. Ať již jako jednotlivý balík, nebo formou repozitáře třetí strany. Chcete \emph{Google Chrome}? Budete ho mít. I~bez nich ale v~repozitářích Fedory naleznete takřka dvacet tisíc balíků obsahujících nejrůznější aplikace a knihovny! Není to ale jen o~licenční čistotě. Fedora důsledně ctí patentové právo v~oblasti software (jakkoliv je v~rámci~EU méně relevantní). Při nasazení Fedory v~komerčním i jiném prostředí máte jistotu, že jednáte v~souladu se zákonem.

\section*{Bezpečnost}
Vývoj Fedory má svá jasná pravidla a bezpečnost je priorita. Ve Fedoře probíhá testování, jako u~kteréhokoli jiného významného softwarového produktu. A~navíc, Fedora se každých několik let stává základem, na kterém staví firma Red Hat svou linuxovou distribuci s~komerční podporou~-- Red Hat Enterprise Linux (RHEL). Stejná firma také platí mnoho vývojářů do práce na Fedoře zapojených a vlastní ochranné známky s~Fedorou související. Podpora Fedory, která se počítá od data vydání aktuální verze distribuce po vydání další plus měsíc (což při dodržení šesti měsíčního cyklu činní třináct měsíců) pak důsledně zahrnuje bezpečnostní záplaty. A~nejen to. Po celou dobu podpory jsou do Fedory začleňovány nové verze jádra. Uplatníme-li hrubou zkratku, že nová verze jádra se rovná novému podporovanému hardware, máme tu další plus bod. Fedora zároveň neobsahuje žádné programy, které by bez vašeho vědomí odesílaly jakékoliv informace. Nikdy. Fedora vás nešmíruje, naopak maximálně ctí vaše soukromí.

\section*{V~čele peletonu}
Do Fedory se dostává velké množství nového software, často je tak Fedora první (nebo jedna z~prvních) distribucí, která daný software nasadí. Fedora je běžně místem, jakýmsi podhoubím, kde k~vývoji takových programů dochází. Otevřený software je o~spolupráci, lidé zapojení do Fedory se často intenzivně věnují i práci mimo distribuci samotnou. Nečekají až někdo něco vyvine a oni to pak budou moci zahrnout do Fedory, ale aktivně se podílí na vývoji přímo u~zdroje, přímo u~konkrétního projektu, ať už spolu s~dalšími vývojáři primárně pracujícími na jiných distribucích, nebo s~dalšími zcela nezávislými vývojáři. Časem se takové programy běžně stávají standardy ve většině linuxového světa. Uživatel Fedory tak jde s~dobou, nebo i má předstih. Samozřejmě tu a tam se uživatel může dostat i k~aplikaci, která ještě není cílena do produkčního prostředí. Ale taková Fedora je. Progresivní. Inovativní. V~čele.

